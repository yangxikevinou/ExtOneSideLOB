\documentclass[openany,oneside]{article}
\usepackage{fullpage}
\usepackage{amsmath,amsthm,amssymb,mathtools,tikz,epsfig,enumerate}
\usepackage{hyperref,titling,titlesec,pdfpages,setspace,fancyhdr,multicol,appendix}
\usepackage[numbers]{natbib}

% formatting
\setlength{\parskip}{2ex}
\setlength{\parindent}{0pt}

% statement environment
\newtheorem{thm}{Theorem}[section]
\newtheorem{prop}[thm]{Proposition}
\newtheorem{lem}[thm]{Lemma}
\newtheorem{cor}[thm]{Corollary}
 
\theoremstyle{definition}
\newtheorem{defn}[thm]{Definition}
\newtheorem{prob}[thm]{Problem}
\newtheorem{eg}[thm]{Example}
 
\theoremstyle{remark}
\newtheorem{rem}[thm]{Remark}

% shorthand notations
\newcommand{\E}{\mathbb{E}} % expectation
\renewcommand{\P}{\mathbb{P}} % probability
\newcommand{\I}{\mathbb{I}} % indicator
\DeclarePairedDelimiter{\abs}{\lvert}{\rvert} % absolute value, or cardinality
\DeclarePairedDelimiter{\norm}{\lVert}{\rVert} % norm


% info
\title{ExtOneSideLOB}
\author{Yangxi Ou \\* Department of Mathematical Sciences \\* Carnegie Mellon University}
\date{Summer 2017}
\hypersetup{bookmarksnumbered=true, 
			bookmarksopen=true,
            unicode=true,
            pdftitle=\thetitle,
            pdfauthor=Yangxi Ou,
            pdfstartview=FitH,
            pdfpagemode=UseOutlines}
%%%%%%%%%%%%%%%%%%%%%%%%%%%%%



\begin{document}

% title and table of contents
\begin{center}	
	\textbf{\Large Generalizations of OPTIMAL EXECUTION IN A GENERAL ONE-SIDED LIMIT-ORDER BOOK by Predoiu, Shaikhet, and Shreve, 2011} \\*[5ex]    
    Comments and research ideas \\*[5ex]
	\theauthor \\*[5ex]
    \thedate \\*[5ex]
	\tableofcontents
\end{center}


% Summary of original and directions of generalizations
\section*{Summary of the original paper, and directions of generalizations}
The following is a brief summary of the original paper OPTIMAL EXECUTION IN A GENERAL ONE-SIDED LIMIT-ORDER BOOK by Predoiu, Shaikhet, and Shreve, published on SIAM J. Finan. Math. in 2011\citep{predoiu2011optimal}. In this entire draft, if not further explained, the phrase ``the paper'', ``the original paper'', and etc. refers to \citet{predoiu2011optimal}.

\begin{itemize}
\item Setup

A single large investor buying a specified security with only market buy orders against a centralized limit order book. His holding of the security is described by $X=\{X_t\}_{t\in[0,T]}$, an non-decreasing cadlag function adapted to a given filtered probability space satisfying the usual conditions $(\Omega, \mathcal{F}, \{\mathcal{F}_t\}_{t\in[0,T]}, \P)$.


\item Input

\begin{itemize}
\item Target quantity to be purchased: $\bar{X} > 0$

\item Time span of purchase: $[0,T]$

\item Hypothetical (equilibrium) best ask price: $A=\{A_t\}_{t\in[0,T]}$, a continuous non-negative $\mathcal{H}^1$ martingale, i.e. $\E[A^\ast_T] < \infty$, where $A^\ast_t := \sup_{0\le s\le t} \abs{A_s}$. $A$ represents the hypothetical (equilibrium) best ask price as if there was zero trading activities from the beginning of the period. It is a quantitative description of the position of the ask side of the limit order book.

\item Hypothetical (equilibrium) shape of limit order book: $F:\mathbb{R}_+ \to \mathbb{R}_+$, an non-decreasing caglad function specifying the cumulative quantity $F(x)$ of securities immediately available for purchase below the price level $x+A_t$ as if there was zero trading activity from the beginning of the period. We can imagine a entire static limit order book pegging to the hypothetical best ask price if there is no trading at all. More technically, if one were to apply PCA to the dynamics of the limit order book in the absence of arriving market buy orders, there is only one component (accounting for $100\%$ of explanatory power): the hypothetical best ask price.

\item Resiliency function: $h:\mathbb{R}_+ \to \mathbb{R}_+$, a strictly increasing locally Lipschitz function, vanishing at the origin ($h(0)=0$) with limit at infinity greater than $\bar{X}/T$ ($h(+\infty) > \bar{X}/T$), representing the rate of replenishment $h(x)$ of limit sell orders at the quantity level $x$. It specifies the response of the ask side of the limit order book to the aggregate market buy order flow: $E_t = X_t - \int_0^t h(E_s) ds, \forall t\in[0,T]$, where $X_t$ is the cumulative market buy orders up to time $t$, and $E_t$ is the actual or observed quantity level in the equilibrium limit order book.

\item Cost functional of purchase: $C(X) := \int_0^T (A_t + D_t) d X^c_t + \sum_{0\le t\le T} [A_t \Delta X_t + \Phi(E_t) - \Phi(E_{t^-})]$, where $D_t$ is the extra marginal cost of continuous purchase at time $t$ (due to walking through the book) and $\Phi:\mathbb{R}_+ \to \mathbb{R}_+$ describes the total cost $\Phi(x)$ of purchasing $x$ amount of security immediately from the hypothetical best ask price in the equilibrium limit order book.
\end{itemize}


\item Output

Optimal execution strategy and expected cost of purchase: $X^\ast:[0,T] \to \mathbb{R}_+$ that minimizes the expected cost $\E C(X)$ among all non-decreasing cadlag adapted process $X$ satisfying $X_T = \bar{X}$.


\item Conclusion

\begin{itemize}
\item WLOG, one can consider deterministic strategies for optimal execution.

\item Optimal execution can be achieved by Type B strategies: an initial lump sum purchase at time $0$, followed by continuous purchase exactly offsetting the resiliency so as to keep a constant level in the equilibrium book, then another lump sum purchase at an intermediate time, again followed by continuous purchase exactly offsetting the resiliency, and finally a terminal lump sum purchase at time $T$.

\item Lump sum purchase could be zero. If the intermediate lump sum purchase is zero, it is called a Type A strategy, a special case of Type B strategy. A characterization regarding convexity is given for the optimality of Type A strategies, i.e. absence of intermediate lump sum in at least one optimal strategy. Uniqueness of optimal strategy is also discussed.
\end{itemize}


\item Methods/techniques
\begin{itemize}
\item Well-posedness (excluding continuity) of the imposed order book dynamics: The equation $E_t = X_t - \int_0^t h(E_s) ds, \forall t\in[0,T]$ has a unique solution $E$ for any admissible $X$. This is a standard ODE argument.

\item Connections between price level, quantity level, are total cost in the equilibrium limit order book: $\psi:\mathbb{R}_+ \to \mathbb{R}_+$ gives the marginal price level $\psi(x)$ corresponding to the quantity level $x$, $\Phi:\mathbb{R}_+ \to \mathbb{R}_+$ gives the total cost $\Phi(x)$ of all securities up to the quantity level $x$ in the equilibrium book. This is mainly bookkeeping, introducing the correct definition/symbols and change of variables.

\item Restatement of the problem: separate the random hypothetical best ask price $A$ (position of the limit order book) and the deterministic shape of the book $\Phi$ and dynamics of the book $h$. Rewrite the cost functional as a function of the \emph{state} $E$ instead of the \emph{control} $X$. The cost function is thus $C(X) = C(E) := \Phi(E_T) + \int_0^T \psi(E_s) h(E_s) ds$.

\item Solving the restated problem using convex analysis: write the cost functional as the sum of a convex function of $E_T$ and a convex function of $h(E_s)$: $C(E) = \Phi(E_T) + \int_0^T g(h(E_s)) ds$, where $g(y) := y\psi(h^{-1}(y))$. Apply convex analysis (and duality). The cost functional is relaxed to the lower semicontinuous convex envelope (convex bi-dual) if it is not convex, in particular, $g$ is replaced by $\hat{g}:=g^{\ast\ast}$. Type A strategy solution is obtained in the convex case. Type B strategy solution is obtained in the relaxed case. Young's measure (convex combinations of states, often obtained as solutions to relaxed convex functionals in calculus of variations) is used to produce optimal solution in the relaxed case.
\end{itemize}

\end{itemize}


The following is my personal comments on the paper.
\begin{itemize}
\item Pros

\begin{itemize}
\item A general enough setup yet a simple explicit solution.
\item (Tedious) technicalities are addressed in details, laying solid ground work for further generalizations. \item Stylized computable examples covering all cases discussed in the paper.
\end{itemize}

\item Cons

\begin{itemize}
\item The perspective is too classic: the emphasis of the paper is on the \emph{form} of the solution instead of essential \emph{properties} of the problem and solutions. In particular, problems with varying input parameters are treated as standalone problems, while continuity of solutions with respect to data is not studied completely. The classical approach usually only permits symbolic computations (under the condition that one can compute subgradients, derivatives and solve equations relatively easily), rather than more general and efficient numerical calculations. (This case is quite unusual numerically: because Type B strategy is finite-dimensional (dimension $6$), one \emph{reduces} an infinite dimensional optimization problem to a finite dimensional one.)
\item The simplified assumptions of deterministic time-homogeneous (morally a notion of ``equilibrium'' or ``semi-static'') shape of the book $F$ and resiliency $h$ might be good approximations to reality, but require more theoretical and empirical scrutiny.
\end{itemize}

\end{itemize}


The following are possible directions of generalizations or complements.
\begin{itemize}
\item Abstract properties and robustness of the problem: continuity/regularity of solutions with respect to inputs (under some appropriate topologies)
\item Time-inhomogeneous resiliency function $h$
\item Not increasing resiliency function $h$
\item Noise order flows in addition to the large investor's orders
\item Price-dependent deterministic resiliency function $h$ (it has been ``considered'' according to the paper)
\item Stochastic resiliency
\item Time-inhomogeneous deterministic equilibrium shape of the limit order book
\item Stochastic shape of the limit order book
\item Efficient numerical methods to compute (almost) minimizers, echoing abstract properties of the problem
\item Incorporate limit buy orders, and model market sell order flows
\end{itemize}


% jumps
\section{Introducing jumps into the best ask price $A$}
In the original model, the hypothetical best ask price $A$ is assumed to be continuous. Continuity can be relaxed to cadlag (but still $A\in \mathcal{H}^1$), the most general possible stochastic integrator, due to the stochastic integration by parts formula. The decomposition of cost functional into deterministic and martingale parts still works under the new assumption, as follows:
\begin{align*}
\int_{[0,T]} A_{t^-} d X_t &= A_T X_T - A_{0^-} X_{0^-} - \int_{[0,T]} X_{t^-} d A_{t} - \sum_{0\le t\le T} \Delta X_t \Delta A_t \quad \textrm{(integration by parts for cadlag semimartingales)} \\
\int_{[0,T]} A_{t} d X_t &= A_T X_T - A_{0^-} X_{0^-} - \int_{[0,T]} X_{t^-} d A_{t} \quad \textrm{(similar to the first formula in Section 3 Problem simplifications)}
\end{align*}
Following exactly the same argument in the original paper (due to $\mathcal{H}^1$ integrability of $A$), we still obtain Equation (3.1): $\E C(X) = \E \int_0^T D_t d X^c_t + \E \sum_{0\le t\le T} [\Phi(E_t) - \Phi(E_{t^-})] + \bar{X} A_{0^-}$. Thus, WLOG, we may assume $A\equiv 0$ even though the actual best ask price can possibly jump.

There is another subtle technical point regarding the portfolio holding process $X$ when $A$ is allowed to have jumps. In the original paper, $X$ is assumed to be cadlag adapted when $A$ is assumed to be continuous. Since the only randomness in the model is due to the \emph{continuous} ask price process $A$, one may assume WLOG (or argue by Occam's razor) that the filtration $\{\mathcal{F}_t\}_{t\in[0,T]}$ is the augmented natural filtration generated by $A$. It is well known\citep{protter2005stochastic} that in such a filtration, the predictable sigma-algebra coincides with the optional sigma-algebra (while the former is in general a subset of the latter), and hence $X$ is indeed predictable. In other words, it does not matter whether $X$ is cadlag adapted or cadlag predictable in this case. However, predictability and optionality is different when $A$ is allowed to have jumps. In finance, portfolio holdings should be assumed \emph{predictable} so as to model the reality of no future information. One should impose that $X$ is cadlag predictable or even caglad adapted if $A$ is a cadlag $\mathcal{H}^1$ martingale. Impose predictability of $X$ does not change the integration by parts formula above, and hence the rest still holds.

In short, one can relax the original model to the following setup easily (without changing the solution):
\begin{itemize}
\item The hypothetical best ask price $X$ is a \emph{cadlag} (nonnegative) $\mathcal{H}^1$ martingale.
\item The investor's purchasing strategy, or portfolio holding $X$ is non-decreasing, cadlag, and \emph{predictable}.
\end{itemize}

The economic implication of this relaxation is to allow surprises in the best ask price, or a sudden unexpected shift in the position of the ask side of the limit order book.


% not increasing resiliency
\section{Not increasing resiliency function $h$ and continuity of the problem}
The original paper assume that the resiliency function $h:\mathbb{R}_+ \to \mathbb{R}_+$ is \emph{strictly} increasing, locally Lipschitz and $h(0) = 0, h(+\infty) := \lim_{x\to +\infty} h(x) > \bar{X}/T$. Local Lipschitzness is necessary for the unique existence of the so-called volume-effect process $E$ (defined via ODE), which describes the quantity level of the book at which the actual/observed best ask price is. The condition $h(0)=0$ incorporates the notion of hypothetical equilibrium \emph{best} ask price, simply saying that there is no competition of liquidity provision at the best ask price level. The other two conditions are reasonable, but more for technical convenience rather than as necessary. We consider some relaxations of the monotonicity and the limit.


\subsection{Relaxing the limit: inadequate resiliency function $h$}
Let's first re-examine the condition that $h(+\infty) > \bar{X}/T$. Economically, $\bar{X}/T$ is the (minimal) average speed of purchase of the large investor. So the limiting condition requires that the rate of resiliency eventually exceeds the average rate of purchase, should a sufficient large quantity in the book has been consumed. We interpret this condition as \emph{adequate} resiliency. I believe that the adequate resiliency condition is artificial (no compelling economic reason and can easily be violated)  and can be removed without changing the solution to the problem (of course, the stronger the resiliency, i.e. more liquidity provision, the better for the large investor). Here is an extreme case: suppose that $h\equiv 0$, in other words, there is no resiliency at all, then any purchasing strategy will give the same cost $\Phi(\bar{X})$, and in particular, an initial discrete lump sum order will do the job, which is still a Type A strategy defined in the paper. We give a formal argument below.

Suppose that $h:\mathbb{R}_+ \to \mathbb{R}_+$ is strictly increasing, locally Lipschitz and $h(0)=0$, but $h(+\infty)$ could take any value in $[0,+\infty]$, i.e. resiliency may be \emph{inadequate}. The original paper does not reference the adequate resiliency condition until Section 4 Equation (4.2), which defines the (restricted compact) domain of $g$ to be $\left[0, \bar{Y}:=\max\{h(\bar{X}), \bar{X}/T\} \right]$. I think that maximum should really be minimum in Subsection 4.1 where $g$ is restricted to be convex. This is due to Equation (4.10) the (unique) $\bar{e} \in (0,\bar{X})$ such that $k(\bar{e}) = \bar{e} + h(\bar{e}) T = \bar{X}$. Note that we have another upper bound for $\bar{e}$: if $h(e) = \bar{X}/T >0$, then $e>0$ and $e+h(e)T > \bar{X}$. Thus $\bar{e} < h^{-1}(\bar{X}/T)$ as well. So $\bar{e} \in \left(0, \min\{\bar{X}, h^{-1}(\bar{X}/T)\} \right)$. Therefore, one only needs to consider $y\in \left[0, \underline{Y}:=\min\{h(\bar{X}), \bar{X}/T\} \right]$ for cost minimization in Subsection 4.1. This point is ``re-confirmed'' in the proof of Theorem 4.2, in which the global Type A strategy minimizer is characterized by some $e^\ast \in [\bar{e}, \bar{X}]$. Hence, we have $h\left(\frac{X-e^\ast}{T}\right) \le h\left(\frac{X-\bar{e}}{T}\right) = h(\bar{e}) < \underline{Y}$, which says that it is enough to restrict the domain of $g$ to $[0,\underline{Y}] \subseteq [0,\bar{Y}]$.

In Subsection 4.2, we can no longer restrict the domain of $g$ to be $[0,\underline{Y}]$ even though it is still true that $e^\ast \in [0,\underline{Y}]$. The issue here is that we study the relaxed cost functional $\hat{C}(X) := \Phi(E_T) + \int_0^T \hat{g}(h(E_t)) dt$ instead of the original cost $C(X)$ ($\hat{g}$ is the lower semicontinuous convex envelope of $g$ over $[0,\bar{Y}]$), and the optimal rate of continuous purchase $y^\ast := \frac{\bar{X}-e^\ast}{T}$ is ``synthesized'' by a convex combination (expectation under a certain probability called Young's measure) of a pair of continuous purchasing rates $\alpha$ and $\beta$, at both of which $g$ and $\hat{g}$ coincide. One cannot guarantee that $\beta \in [0,\underline{Y}] \subseteq [0,\bar{X}/T]$ any more, but $\beta \le h(\bar{X})$ still holds. An optimal Type B strategy $X^B$ (which does not degenerate to Type A) has the following form, for some carefully chosen $t_0 \in [0,T]$:
\begin{align*}
X^B_t =
\begin{cases}
h^{-1}(\alpha) + \alpha t, & \quad 0\le t < t_0, \\
h^{-1}(\beta) + \alpha t_0 + \beta (t-t_0), & \quad 0\le t < T, \\
\bar{X}, & \quad t=T
\end{cases}
\end{align*}
Inequality (4.25) along its proof shows that such $X^B$ is non-decreasing, and in particular, $h^{-1}(\beta) \le X^B_{t_0} \le X^B_{T} = \bar{X} \Rightarrow \beta \le h(\bar{X})$ as claimed. So one can further restrict the domain of $g$ to $[0,Y:=h(\bar{X})]$ in general, instead of the larger interval $[0,\bar{Y}]$. Even though the relaxed function $\hat{g}$ depends on the domain of $g$ at a first glance, its value does not change for those points where $g$ and $\hat{g}$ coincide if we shrink the domain. In particular, if we make the new domain to be $[0,Y]$, we still obtain the same optimal Type B strategy with the same $t_0$, $\alpha$ and $\beta$.

In summary, we can drop the adequate resiliency condition and make only the following modifications without change our reasoning and conclusions:
\begin{itemize}
\item $h:\mathbb{R}_+ \to \mathbb{R}_+$ is strictly increasing, locally Lipschitz, and $h(0)=0$.
\item The domain of $g$ is restricted to the interval $[0,h(\bar{X})]$.
\end{itemize}

In terms of economics, resiliency need not be too large even when a large volume of the book has been consumed, and the large investor should consider a smaller interval of resiliency rate (and hence continuous purchasing rate) when searching for optimal execution strategy. In a nutshell, if he pushes the resiliency rate above $h(\bar{X})$, he has already achieved his target quantity $\bar{X}$ no matter his path of purchasing. The optimal strategy takes the exact same form as before.


\subsection{Not increasing resiliency function $h$}
Now, further assume that $h$ is merely increasing but not necessary strict while keeping local Lipschitzness and $h(0)=0$ intact. In terms of economics, competition for liquidity provision is not intensified as more quantities of securities are consumed. This can reasonably happen if \emph{all} liquidity providers increase rate of replenishment only when the actual best ask price changes.

Under the new assumption the quantity level $E_t = X_t - \int_0^t h(E_s) ds$ is still well-defined. The deductions in Section 2 and 3 are unchanged verbatim, in particular, we still have the restated simplified cost functional: $C(X) = \Phi(E_T) + \int_0^T \psi(E_t) h(E_t) dt$. The first key step in Section 4, definition of the function $g(y):= y\psi(h^{-1}(y))$, now requires attention as $h$ is no longer invertible. We mitigate the problem by using the lower-continuous (left-continuous) inverse $\underline{h}^{-1}$ of $h$, where $\underline{h}^{-1}(y) := \inf\{x\in\mathbb{R}_+ : h(x)\ge y\}$. (This is very intuitive: for the same rate of resiliency, i.e. marginal liquidity provision, the investor would like to trade at a lower level of the book to obtain a marginal price no higher.) Note that $h$ is actually continuous and hence surjective onto $[0,h(\bar{X})]$ if its domain is restricted to $[0,\bar{X}]$ (since $0\le E_t \le \bar{X}, \forall t\in[0,T]$), so we have $\underline{h}^{-1}(y) = \inf\{x\in\mathbb{R}_+ : h(x)=y\} = \min\{x\in\mathbb{R}_+ : h(x)=y\}$, and hence $h\left(\underline{h}^{-1}(y)\right) = y$ and $\underline{h}^{-1}(h(x)) \le x$. In other words, $\underline{h}^{-1}$ is the smallest right inverse of $h$, and it is increasing as well. With this relation in mind, we define $g(y):=y\psi(\underline{h}^{-1}(y))$ in place of Equation (4.1), and the following three observations (4.3), (4.4), and (4.5) hold with Equation (4.4) becoming
\begin{align*}
C(X) = \Phi(E_T) + \int_0^T \psi(E_t) h(E_t) dt \ge \Phi(E_T) + \int_0^T \psi(\underline{h}^{-1}(h(E_t))) h(E_t) dt = \Phi(E_T) + \int_0^T g(h(E_t)) dt
\end{align*}
with equality if $E_t \in \underline{h}^{-1}([0,h(\bar{X})])$, for $t\in[0,T]$ a.e. (only sufficient, as $\psi$ itself is not strictly increasing).

Continuing to Subsection 4.1, here we first assume that $g$ defined above is convex. Remark 4.1 is the same as in the original paper, as $k$ is continuous and strictly increasing because $k(e) = e + h(e) T$. In particular, there is a unique $\bar{e} \in [0,\bar{X}]$ such that $k(\bar{e}) = \bar{e} + h(\bar{e}) T = \bar{X}$. Thus, the admissibility condition of a Type A strategy is still $\bar{e} \le E^A_T \le \bar{X}$. For the proof of Theorem 4.2, there could be two alternative routes. The first one is to follow almost the same arguments as in the paper with $h^{-1}$ replaced by $\underline{h}^{-1}$. The slight difference one make here is to optimize $G(e) := \Phi(e) + T g\left(\frac{\bar{X}-e}{T}\right)$ over the admissible interval $[\bar{e}, \bar{X}]$ instead of $[0,\bar{X}]$. (This route can be seen as a simplification of the original proof of Theorem 4.2 in the paper, so that it is unnecessary to check admissibility again.) However, one cannot conclude that the global minimizer of $G$ is a critical point (in terms of existence of a zero-subdifferential) as implied by the original proof via this simplification. To obtain this stronger implication, one need to follow Case I in the proof of Theorem 4.5 and consider $e^\ast$ to be the \emph{greatest} minimizer of $G$ (or $\hat{G}$). This is the second route to show Theorem 4.2 under the generalization that $h$ is non-decreasing. Either way, the optimal strategy has $X^A_0 = E^A_t = \underline{h}^{-1}\left(\frac{\bar{X}-e^\ast}{T}\right), \forall t\in[0,T)$, which makes the above Inequality (4.4) an equality. Similarly, replacing $h^{-1}$ with $\underline{h}^{-1}$ in the proof of Theorem 4.5 gives us the exact same conclusion in Subsection 4.2 where $g$ need not be convex. Thanks to the right inverse property that $h\circ \underline{h}^{-1} = id$ when restricted to $[0,h(\bar{X})]$. Inequality (4.4) above also becomes an equality.

Thus, one can relax strictly increasing $h$ to non-decreasing $h$ in the original setup, and the conclusion remains the same. The key here is to use \emph{lower semicontinuous} inverse $\underline{h}^{-1}$, characterized as the smallest right inverse of $h$.

To ``go beyond'' monotonicity, i.e. $h$ is only assumed to be locally Lipschitz and $h(0)=0$, we convert the situation to the monotone case by introducing the running maximum $\bar{h}:\mathbb{R}_+ \to \mathbb{R}_+$ of $h$, given by $\bar{h}(x) := \sup_{0\le z\le x} h(z)$, characterized as the smallest non-decreasing function above $h$. Recall or observe that $\underline{h}^{-1}$ is (still) the smallest right inverse of $h$. To be clear and connect all functions derived from $h$, we state the following lemma.
\begin{lem}
Let $f:\mathbb{R}_+ \to \mathbb{R}_+$ be a continuous non-negative function defined on the half real line such that $f(0)=0$. Set $\bar{f}:\mathbb{R}_+ \to \mathbb{R}_+$ to be $\bar{f}(x) := \sup_{0\le z\le x} f(z)$ and $\underline{f}^{-1}:f(\mathbb{R}_+) \to \mathbb{R}_+$ to be $\underline{f}^{-1}(y) := \inf\{x\in\mathbb{R_+} : f(x) \ge y\}$. Then the followings hold:
\begin{enumerate}
\item $\bar{f}$ is continuous, non-decreasing, with $\bar{f}(0)=0$, $\bar{f}(\mathbb{R}_+)  = f(\mathbb{R}_+)$ and $\bar{f} \ge f$;
\item If $\tilde{f}$ is non-decreasing and $\tilde{f} \ge f$, then $\tilde{f} \ge \bar{f}$;
\item If $x\in \mathbb{R}_+$ is such that $\bar{f}(x) > f(x)$, then there is $z\in(0,x)$ such that $\bar{f}(z) = f(z)$;
\item $\underline{f}^{-1}$ is caglad, non-decreasing, with $\underline{f}^{-1}(0)=0$, $f\circ \underline{f}^{-1}(y) = y, \forall y\in f(\mathbb{R}_+)$ and $\underline{f}^{-1} \circ f (x) \le x, \forall x\in \mathbb{R}_+$ with equality iff $x\in \underline{f}^{-1}(f(\mathbb{R}_+))$;
\item If $k:f(\mathbb{R}_+) \to \mathbb{R}_+$ is a right inverse of $f$, i.e. $f\circ k = id$, then $\underline{f}^{-1} \le k$;
\item $\underline{\left(\bar{f}\right)}^{-1}$ is a right inverse of $f$, and hence $\underline{\left(\bar{f}\right)}^{-1} = \underline{f}^{-1}$.
\end{enumerate}
\end{lem}

Leveraging on the special case of non-decreasing $h$ above, we can tackle non-monotone $h$ easily by considering $\bar{h}$. Inequality (4.4) $C(X) = \Phi(E_T) + \int_0^T \psi(E_t) h(E_t) dt \ge \Phi(E_T) + \int_0^T g(h(E_t)) dt$ remains true, and the equality is attained when $E_t \in \underline{h}^{-1}([0,\bar{h}(\bar{X})])$ for a.e. $t\in[0,T]$. But for such $E_t$, we have $h(E_t) = \bar{h}(E_t)$, and hence we can minimize the cost functional with $\bar{h}$ instead. In Remark 4.1, a Type A strategy will be admissible if $E^A_T \in[\bar{e},\bar{X}]$, where $\bar{e}$ satisfies $\bar{k}(\bar{e}) = \bar{X}$ and $\bar{k}(e) := e + \bar{h}(e) T$. Such $\bar{e} \in [0,\bar{X}]$ is well-defined since $\bar{k}$ is continuous and strictly increasing. The proof of Theorem 4.2 and Theorem 4.5 follow the special case that $h$ is non-decreasing.

To end this subsection, we can impose the following less stringent conditions on the resiliency function $h$ while keeping the original conclusions:
\begin{itemize}
\item $h(0)=0$
\item $h$ is locally Lipschitz
\end{itemize}
The financial intuition of the solution is straightforward: for the same level of marginal liquidity supply (i.e. resiliency) which is entirely determined by the quantity level of the book, the investor would select the lowest possible quantity level (whose mathematical counterpart is lower-semicontinuous inverse) so as to minimize margin execution price.


% time-dependent resiliency
\section{Time-dependent resiliency function $h$}
So far, we have relaxed the model to include:
\begin{itemize}
\item $A$ is a cadlag (nonnegtaive) $\mathcal{H}^1$ martingale;
\item $X$ is non-decreasing, cadlag and predictable;
\item $h$ is locally Lipschitz and $h(0)=0$.
\end{itemize}

The next natural generalization is to consider time-inhomogeneous resiliency $h:[0,T]\times \mathbb{R}_+ \to \mathbb{R}_+$, with $h(t,x)$ representing the replenishment speed at time $t$ if the book is at the quantity level $x$. It is immediate to see that we should require $h(t,\cdot)$ to be locally Lipschitz and $h(t,0)=0$ for $t\in[0,T]$ (or weaker, a.e.). However, it is not clear what regularity condition we should impose for the time variable $t$, i.e. for the function $t\mapsto h(t,\cdot)$, which describes the evolution of the resiliency function. We delay the technicalities for the time being, and informally discuss alternate solution techniques for the original problem. This will help us better understand the time-inhomogeneous case.

\subsection{A digression to calculus of variations: Lagrangian, Hamiltonian, and duality}
This subsection is to introduce Euler-Lagrangian equation, Hamilton-Jacobi-Bellman equation, and dual function for the original problem (possibly with time-inhomogeneous resiliency). We focus on the big pictures and omit details at the expense of mathematical rigor. In addition, we don't claim that any standalone method could completely solve the original problem.

Suppose that resiliency function $h:[0,T]\times \mathbb{R}_+ \to \mathbb{R}_+$ is time-dependent. We write $h_t$ for $h(t,\cdot)$ for notational ease, and could define $\overline{h_t}$ and $\underline{h_t}^{-1}$ as before. Moreover, we can set $g_t(y) := y \psi(\underline{h_t}^{-1}(y))$ for $y\in [0,\overline{h_t}(\bar{X})]$ and $t\in[0,T]$. Assume necessary regularity (continuity or differentiability) for functions whenever needed in the following computations.

The first way to minimize the functional $C(X) = \Phi(E_T) + \int_0^T g_t(h_t(E_t)) dt$ is to consider variations. Suppose that a minimizer is known, any perturbed admissible strategy will cost more than the minimizer. This idea leads to the Euler-Lagrange equation, which is essential a critical point condition for the cost functional at the minimizer. Symbolically, let $E$ be a minimizer and $\delta E$ be any admissible direction of variation in the tangent space of plausible volume-effect processes, then the critical point condition is given by
\begin{align*}
0 =& \frac{dC}{dE}(\delta E) = \partial \Phi(E_T) \delta E_T + \int_0^T \partial g_t(h_t(E_t)) \partial h_t (E_t) \delta E_t dt \\
=& \partial \Phi(E_T) \left(\delta X_T - \int_0^T \partial h_t(E_t) \delta E_t dt \right) + \int_0^T \partial g_t(h_t(E_t)) \partial h_t (E_t) \delta E_t dt = \int_0^T (\partial g_t(h_t(E_t)) - \partial \Phi(E_T))\partial h_t (E_t) \delta E_t dt
\end{align*}
In the computation above, we invoked the constraint that $X_T \equiv \bar{X}$ so $\delta X_T \equiv 0$. In ``nice'' cases, this requires that $\partial g_t(h_t(E_t)) - \partial \Phi(E_T)$ for a.e. $t\in [0,T]$, i.e. $\partial g_t(h_t(E_t))$ is a constant independent of time $t$. Notice that if $h$ is time-homogeneous, then $\partial g(h(E_t))$ is a constant means that $E_t$ is a constant, which reduces to a Type A strategy in the original paper. Of course, a Type B strategy does not satisfy this constant $E$ condition, but it has a piecewise constant $E$. To mitigate this glitch, one needs to consider distributional derivatives instead of classical ones. Nonetheless, the Euler-Lagrange equation does gives us hints of the optimum. Other than the issue of regularity of functions, one do have to keep in mind that the Euler-Lagrange equation only specifies critical points (if interpreted classically), but minimizers are not necessarily critical points and critical points are not necessarily minimizers. Both coincides only at ``special'' cases, for example, Subsection 4.1 in the original paper, in which a Type A strategy is both a minimizer and a critical point of the cost functional.

The second way to minimize the functional $C$ is dynamic programming. Instead of considering a single minimization problem, dynamic programming studies a family of problems parametrized by state variables, connects all these problems using the dynamic programming principle, and derives a PDE called Hamilton-Jacobi-Bellman (HJB) equation satisfied by the minimal values of parametrized problems. We use the time $t$, the quantity level of the book $e$, and remaining quantity $x$ to purchase as state variables. Let $v(t,e,x)$ stands for the \emph{minimal} purchasing cost of $x$ \emph{additional} shares when the time is $t$ and the book is at level $e$, i.e. $v(t,e,x) := \inf \left\{ \Phi(E_T) - \Phi(e) + \int_t^T g_s(h_s(E_s)) ds : E_{t} = e, E_T - e + \int_t^T h_s(E_s) ds = x \right\}$. The minimal $C(X)$ we are looking for is $v({0^-},0,\bar{X})$ (as we have $E_{0^-}=0$ to allow possible jumps at time $0$). It is clear by definition that $v(T,e,x)=0$ if $x\le 0$ and $v(T,e,x)=+\infty$ if $x>0$. For other parametrized minimizations, dynamic programming principle proposes that the Hamiltonian $v(t,E_{t},\bar{X}-X_{t}) + \Phi(E_{t}) + \int_0^t g_s(h_s(E_s)) ds$ be a non-decreasing function of time $t$ for all admissible strategies $X$, and be constant in time $t$ for optimal strategy $X$. In particular, for jumps, we have 
\begin{align*}
v(t^-, E_{t^-}, \bar{X}-X_{t^-}) + \Phi(E_{t^-}) &= \inf_{\Delta X_t \ge 0} \left\{ v(t,E_t,\bar{X}-X_t) + \Phi(E_t) \right\} \\
&= \inf_{\Delta X_t \ge 0} \left\{ v(t, E_{t^-}+\Delta X_t, \bar{X}-X_{t^-}-\Delta X_t) + \Phi(E_{t^-} + \Delta X_t) \right\},
\end{align*}
that is $v(t^-,e,x) + \Phi(e) = \inf_{a\ge 0} \left\{v(t,e+a,x-a) + \Phi(e+a) \right\}$. At the instance before terminal time $T$, one must have $v(T^-,e,x)=\Phi(e+x)-\Phi(e)$ if $x>0$ and zero otherwise, as the only way to achieve the purchasing target (avoiding infinite cost) is to buy $a=x>0$ shares immediately. In differential form (ignoring possible jumps), using the equation $X_t = E_t + \int_0^t h_s(E_s) ds$, one has
\begin{align*}
0 &= \partial_t v + g_t(h_t(e)) + \inf_{a\ge 0} \left\{ (\partial \Phi(e) + \partial_e v) (a-h_t(e)) - a \partial_x v \right\} \\
&= \partial_t v + g_t(h_t(e)) + \inf_{a\ge 0} \left\{ (\partial \Phi(e) + \partial_e v - \partial_x v) a - h_t(e) (\partial \Phi(e) + \partial_e v) \right\} = \partial_t v - h_t(e) \partial_e v + g_t(h_t(e)) - h_t(e) \partial \Phi(e),
\end{align*}
since either $a=0$ or $\partial \Phi(e) + \partial_e v - \partial_x v = 0$. To have the infimum well-posed, one must impose that $\partial \Phi(e) + \partial_e v - \partial_x v \ge 0$. If $\partial \Phi(e) + \partial_e v - \partial_x v > 0$, then it is clear that $a=0$, i.e. the rate of purchase is zero at those particular moments. However, if $\partial \Phi(e) + \partial_e v - \partial_x v = 0$, then the HJB equation \emph{alone} does not give us any information about the optimal trading strategy, and we have to infer this piece of information from something else. In general HJB equation is difficult to solve explicitly, but if one has a solution with sufficient regularity in some special case, then a standard verification argument will give us the minimal cost of the problem. Moreover, if one can infer the optimal trading strategy from HJB, then the strategy is usually given in the so-called feedback form, i.e. a function in terms of the state variables. On the contrary, one does not obtain a feedback form strategy from the Euler-Lagrange equation.

Both ways above involve solving differential equations, which rely on some regularities of the solutions and hence some restrictions of the inputs. A promising way to get around the technicalities of differential equations is (convex) duality. In addition to the original constrained optimization problem which is called the primal problem, the method introduces a related problem called the dual problem which plays the same role as Lagrange multiplier in classical calculus. The advantage of duality method is its minimal assumptions on regularities of functions, and all one has to do is to check zero duality gap. Let $(\lambda,\mu)$ be the dual variable, and define $D(\lambda,\mu) := \inf \left\{ \Phi(E_T) + \int_0^T g_s(h_s(E_s)) ds - \lambda \int_0^T h_s(E_s) ds - \mu E_T : X_{0^-} =0, E_t = X_t - \int_0^t h_s(E_s) ds \right\}$. Denote the convex duals of $\Phi$ and $g_s$ by $\Phi^\ast(y):=\sup_{x\ge 0}\left\{xy - \Phi(x)\right\}$ and $g_s^\ast(y):=\sup_{x\ge 0}\left\{xy - g_s(x)\right\}$ respectively. Then for admissible trading strategies $X$, we have
\begin{align*}
C(X) &= \Phi(E_T) + \int_0^T g_s(h_s(E_s)) ds = \Phi(E_T) + \int_0^T g_s(h_s(E_s)) ds - \lambda \int_0^T h_s(E_s) ds - \lambda E_T + \lambda \bar{X} \\
& \ge D(\lambda, \lambda) + \lambda \bar{X} \left( \ge -\Phi^\ast(\lambda) - \int_0^T g_s^\ast(\lambda) ds + \lambda \bar{X} \right)
\end{align*}
Taking supremum over all $\lambda \ge 0$ of the right hand side, we have a lower bound on the minimum trading cost. If there is zero duality gap, i.e. equality could be achieved, then one can figure out the optimal trading strategy by first finding the maximizer $\lambda^\ast$ of $\sup_{\lambda \ge 0}\left\{D(\lambda,\lambda) + \lambda \bar{X} \right\}$, and then the minimizer of $D(\lambda^\ast, \lambda^\ast)$. It seems at first that we complicate the problem by layering over one more optimization, but the dual problem involving $D(\lambda,\lambda)$ is actually far simpler as it is merely one-dimensional. In the case that the resiliency function $h(s,x)$ is time-homogeneous and so is $g_s = g$, the dual function in the parenthesis above can be written down explicitly. A sufficient condition for zero-duality gap is that $g$ is convex. We believe that the original paper was inspired by this duality approach to consider Type A strategy first. It should be noted that in other cases, the issue of duality gap is quite delicate and the one that requires most attention.

Here is a brief review of this subsection before we move on to tackle the general case of time-inhomogeneous resiliency. We have outlined three methods to solve the problem, each with its own convenience and technicalities. Even though these methods may not assemble to a full solution, they would serve as starting points and give us significant inspirations.


\subsection{Separable resiliency and time-homogeneous elasticity: a special case}


\subsection{Critical point minimizers and spoofing (permitting limit sell)}


\subsection{General existence and robustness via abstract properties: continuity and compactness}


\section{Introducing noises to order flows and/or resiliency}

% references
\bibliographystyle{plainnat}
\phantomsection
\addcontentsline{toc}{section}{\refname}
\bibliography{References}

\end{document}